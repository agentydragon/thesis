\chapter{Název první kapitoly}

\section{Název první podkapitoly v první kapitole}

\section{Název druhé podkapitoly v první kapitole}

\subsection{Ukázka}
\label{ssec:ukazka}

\begin{itemize}
\item Logo Matfyzu vidíme na obrázku~\ref{fig:mff}.
\item Tato subsekce má číslo~\ref{ssec:ukazka}.
\end{itemize}

červeno-černý (krátká),
strana 16--22 (střední),
$45-44$ (minus),
a~toto je --- jak se asi dalo čekat --- vložená věta ohraničená dlouhými pomlčkami.
(Všimněte si, že jsme za \verb|a| napsali vlnovku místo mezery: to aby se
tam nemohl rozdělit řádek.)

\newtheorem{theorem}{Věta}
\newtheorem*{define}{Definice}	% Definice nečíslujeme, proto "*"

\begin{define}
{\sl Strom} je souvislý graf bez kružnic.
\end{define}

\begin{theorem}
Tato věta neplatí.
\end{theorem}

\begin{proof}
Neplatné věty nemají důkaz.
\end{proof}

\begin{figure}
	\centering
	\includegraphics[width=30mm]{img/logo.eps}
	\caption{Logo MFF UK}
	\label{fig:mff}
\end{figure}

\begin{itemize}
\item Model externí paměti

	CPU + malá cache + velká externí paměť.
	Zanedbáme všechny cache kromě poslední.

	Block size $B$ (v ext. paměti i v cachi),
	velikost cache $M$, tedy $M/B$ cache lines.

	Registry, L1, L2, RAM, disk, síť, ...
	Typicky L1 cache line = L2 cache line = 64B.
	\cite{Vit}

	\begin{itemize}
	\item Základní datové struktury: queue, stack
	\end{itemize}

\item Co je to slovník, operace
	\begin{itemize}
	\item Příklady použití
	\item Rozšíření: successor, predecessor
		\begin{itemize}
		\item Příklady použití
		\end{itemize}
	\item Rozšíření: multiset
		\begin{itemize}
		\item Příklady použití
		\end{itemize}
	\end{itemize}

\item Triviální implementace slovníku
	\begin{itemize}
	\item Triviální pole (unrolled linked list?)
	\item Triviální přímá adresace pro malé klíče

\item Hashování
	\begin{itemize}
		\item Seznam prvků v kolizi
		\item Linear probing (open addressing)
		\item Double hashing
		\item Cuckoo hashing
	\end{itemize}

\item Vyhledávací stromy
	\begin{itemize}
		\item B-stromy (a 2-3-4 stromy)
		\item Červenočerné stromy
		\item Splay stromy (TODO: zkusit vymyslet N-arni splay stromy)
		\item Van Emde-Boas uspořádání
		\item T-stromy (?)
	\end{itemize}

\item Speciality pro vnější paměť
	\begin{itemize}
		\item Directory hashing (extensible hashing, linear hashing)
		\item Directoryless hashing methods
		\item Tree EM structures
	\end{itemize}


\item Popis prostředí, kde měřím
\end{itemize}
