%%% Hlavní soubor. Zde se definují základní parametry a odkazuje se na ostatní části. %%%

%% Verze pro jednostranný tisk:
% Okraje: levý 40mm, pravý 25mm, horní a dolní 25mm
% (ale pozor, LaTeX si sám přidává 1in)
\documentclass[12pt,a4paper]{report}
\setlength\textwidth{145mm}
\setlength\textheight{247mm}
\setlength\oddsidemargin{15mm}
\setlength\evensidemargin{15mm}
\setlength\topmargin{0mm}
\setlength\headsep{0mm}
\setlength\headheight{0mm}
% \openright zařídí, aby následující text začínal na pravé straně knihy
\let\openright=\clearpage

%% Pokud tiskneme oboustranně:
% \documentclass[12pt,a4paper,twoside,openright]{report}
% \setlength\textwidth{145mm}
% \setlength\textheight{247mm}
% \setlength\oddsidemargin{15mm}
% \setlength\evensidemargin{0mm}
% \setlength\topmargin{0mm}
% \setlength\headsep{0mm}
% \setlength\headheight{0mm}
% \let\openright=\cleardoublepage

%% csLaTeX
\usepackage[czech]{babel}

\usepackage[utf8]{inputenc}

\usepackage{graphicx}
\usepackage{amsthm}
\usepackage{todonotes}

%% Balíček hyperref, kterým jdou vyrábět klikací odkazy v PDF,
%% ale hlavně ho používáme k uložení metadat do PDF (včetně obsahu).
%% POZOR, nezapomeňte vyplnit jméno práce a autora.
%\usepackage[ps2pdf,unicode]{hyperref}   % Musí být za všemi ostatními balíčky
\usepackage[unicode]{hyperref}   % Musí být za všemi ostatními balíčky
\hypersetup{pdftitle=Praktické datové struktury} % TODO: mozna v anglictine?
\hypersetup{pdfauthor=Michael Pokorný}

%%% Drobné úpravy stylu

% Tato makra přesvědčují mírně ošklivým trikem LaTeX, aby hlavičky kapitol
% sázel příčetněji a nevynechával nad nimi spoustu místa. Směle ignorujte.
\makeatletter
\def\@makechapterhead#1{
  {\parindent \z@ \raggedright \normalfont
   \Huge\bfseries \thechapter. #1
   \par\nobreak
   \vskip 20\p@
}}
\def\@makeschapterhead#1{
  {\parindent \z@ \raggedright \normalfont
   \Huge\bfseries #1
   \par\nobreak
   \vskip 20\p@
}}
\makeatother

% Toto makro definuje kapitolu, která není očíslovaná, ale je uvedena v obsahu.
\def\chapwithtoc#1{
\chapter*{#1}
\addcontentsline{toc}{chapter}{#1}
}

\begin{document}

% Trochu volnější nastavení dělení slov, než je default.
\lefthyphenmin=2
\righthyphenmin=2

%%% Titulní strana práce
\pagestyle{empty}
\begin{center}

\large

Univerzita Karlova v Praze

\medskip

Matematicko-fyzikální fakulta

\vfill

{\bf\Large BAKALÁŘSKÁ PRÁCE}

\vfill

\centerline{\mbox{\includegraphics[width=60mm]{img/logo.eps}}}

\vfill
\vspace{5mm}

{\LARGE Michael Pokorný}

\vspace{15mm}

% Název práce přesně podle zadání
{\LARGE\bfseries Název práce} % TODO

\vfill

% Název katedry nebo ústavu, kde byla práce oficiálně zadána
% (dle Organizační struktury MFF UK)
Název katedry nebo ústavu % TODO

\vfill

\begin{tabular}{rl}

Vedoucí diplomové práce: & Mgr. Martin Mareš, PhD.\\
\noalign{\vspace{2mm}}
Studijní program: & informatika \\
\noalign{\vspace{2mm}}
Studijní obor: & obecná informatika \\
\end{tabular}

\vfill

Praha 2014

\end{center}

\newpage

%%% Následuje vevázaný list -- kopie podepsaného "Zadání bakalářské práce".
%%% Toto zadání NENÍ součástí elektronické verze práce, nescanovat.

%%% Na tomto místě mohou být napsána případná poděkování (vedoucímu práce,
%%% konzultantovi, tomu, kdo zapůjčil software, literaturu apod.)

\openright

\noindent
TODO: Poděkování.

\newpage
%%% Strana s čestným prohlášením k bakalářské práci

\vglue 0pt plus 1fill

\noindent
Prohlašuji, že jsem tuto bakalářskou práci vypracoval samostatně a výhradně
s~použitím citovaných pramenů, literatury a dalších odborných zdrojů.

\medskip\noindent
Beru na~vědomí, že se na moji práci vztahují práva a povinnosti vyplývající
ze zákona č. 121/2000 Sb., autorského zákona v~platném znění, zejména skutečnost,
že Univerzita Karlova v Praze má právo na~uzavření licenční smlouvy o~užití této
práce jako školního díla podle §60 odst. 1 autorského zákona.

\vspace{10mm}

\hbox{\hbox to 0.5\hsize{%
V ........ dne ............
\hss}\hbox to 0.5\hsize{%
Podpis autora
\hss}}

\vspace{20mm}


\newpage
%%% Povinná informační strana diplomové práce

\vbox to 0.5\vsize{
\setlength\parindent{0mm}
\setlength\parskip{5mm}

Název práce:
TODO: Název práce % přesně dle zadání

Autor:
Michael Pokorný

Katedra:  % Případně Ústav:
TODO: Název katedry či ústavu, kde byla práce oficiálně zadána % dle Organizační struktury MFF UK

Vedoucí bakalářské práce:
Mgr. Martin Mareš, PhD., Katedra aplikované matematiky

Abstrakt:
TODO
% TODO: v rozsahu 80-200 slov; nejedná se však o opis zadání diplomové práce

Klíčová slova:
TODO % TODO: 3 až 5

\vss}\nobreak\vbox to 0.49\vsize{
\setlength\parindent{0mm}
\setlength\parskip{5mm}

Title:
TODO % přesný překlad názvu práce v angličtině

Author:
Michael Pokorný

Department:
TODO Název katedry či ústavu, kde byla práce oficiálně zadána % dle Organizační struktury MFF UK v angličtině

Supervisor:
Mgr. Martin Mareš, PhD., Department of Applied Mathematics

Abstract:
TODO % v rozsahu 80-200 slov v angličtině; nejedná se však o překlad zadání diplomové práce

Keywords:
TODO % 3 až 5 v angličtině

\vss}


\newpage

%%% Strana s automaticky generovaným obsahem bakalářské práce. U matematických
%%% prací je přípustné, aby seznam tabulek a zkratek, existují-li, byl umístěn
%%% na začátku práce, místo na jejím konci.

\openright
\pagestyle{plain}
\setcounter{page}{1}
\tableofcontents

%%% Jednotlivé kapitoly práce jsou pro přehlednost uloženy v samostatných souborech
\include{uvod}
\chapter{Model externí paměti}

Časová náročnost algoritmů a datových struktur se většinou měří v takzvaném
RAM (\textit{Random Access Memory}) modelu. Paměť RAM modelu je neomezená a
skládá se ze stejně velkých buněk označených čísly od 0 výše. Každá buňka
obsahuje jedno číslo z předem vymezeného intervalu. Program se reprezentuje
jako posloupnost instrukcí, na které se lze odkazovat čísly. Kompletní sada
instrukcí pro RAM se skládá například z:
\begin{itemize}
\item
	Načtení konstanty do buňky: například \texttt{[3]=123} uloží do buňky
	3 hodnotu 123.
\item
	Zkopírování obsahu jedné buňky do druhé: například \texttt{[0]=[1]}
	zkopíruje obsah buňky 1 do buňky 0.
\item
	Dereference: například instrukce \texttt{[[0]]=[1]} uloží obsah
	buňky 1 na místo, které je uložené v buňce 0 a instrukce
	\texttt{[1]=[[0]]} naopak do buňky 1 uloží obsah buňky, jejíž
	číslo je uloženo v buňce 0.
\item
	Aritmetické operace, které vezmou obsahy dvou buněk,
	provedou nad nimi binární operaci a výsledek uloží do jiné buňky.
	Typická sada operací může zahrnovat sčítání, odečítání, násobení,
	celočíselné dělení a operaci zbytku po dělení, bitový AND, OR a XOR a
	bitové posuny doleva a doprava.
	Například \texttt{[4]=[1]+[5]} přečte obsahy buňek 1 a 5, sečte je a
	výsledek uloží do buňky 4.
\item
	Podmíněný skok: například není-li v buňce číslo \texttt{x}
	uložena nula, skoč na instrukci, jejíž číslo je uloženo v buňce
	\texttt{target}. Například \texttt{if [0] 30}: není-li v buňce číslo
	0, skoč na instrukci číslo 30.
\item
	Zastavení programu.
\end{itemize}

Vstup a výstup RAM modelu se může například předávat na předem daných buňkách,
nebo lze pro vstup a výstup přidat další instrukce.

V RAM modelu je dále potřeba specifikovat chování v případě čtení
neinicializované paměti. Obvykle se čtení z neinicializované paměti považuje
za chybu programu a požaduje se, aby se všechna paměť před čtením
vynulovala.
\todo{Zmínit magii, která se stane, když se to umožní.}
Paměťová složitost programu běžícího v RAM se pak měří počtem popsaných buněk,
zatímco časová složitost je dána počtem vykonaných instrukcí.
Jméno \textit{random access memory} připomíná, že tento model na rozdíl od
jiných (například Turingova stroje) umožňuje v konstantním čase přistoupit
na libovolné místo v paměti.

Architektura skutečných počítačů se však od RAM modelu podstatně liší.
Zatímco v RAM modelu se platí konstantní cena za přístup k libovolné buňce
paměti, v reálných počítačích se čas přístupu do paměti výrazně liší
v závislosti na druhu paměti, ve kterém jsou data uložena.
Tabulka \ref{table:speeds} tento fakt ilustruje porovnáním
řádových rychlostí přístupu k různým druhům paměti.

\begin{table}
	\centering
	\begin{tabular}{l|r|l}
	Druh paměti & Velikost & Čas na přístup \\
	\hline
	CPU registry & 128 B & 0.1 ns \\
	L1 cache & 32 kB & 0.5 ns \\
	L2 cache & 256 kB & 7 ns \\
	L3 cache & 3 MB & 30 ns \\
	RAM & 8 GB & 100 ns \\
	Pevný disk & 500 GB & 3 ms \\
	Vzdálený počítač & -- & 50 ms
	\end{tabular}
	\caption{Porovnání rychlostí přístupu k různým druhům paměti}
	\label{table:speeds}
\end{table}
\todo{Najít přesnější čísla}

Instrukce pro skutečné procesory umožňují operace s registry a RAM
\footnote{RAM zde v hardwarovém významu.}.
Registry jsou drobné paměti určené pro \uv{dočasné hodnoty}. Například CPU
Intel Core i7 mají 16 registrů, každý z nich velký 64 bitů.
Operace, které operují s RAM, jsou transparentně zrychleny pomocí hierarchie
\textit{vyrovnávacích pamětí} (anglicky \textit{cache}).
Nejrychlejší a nejmenší cache se označuje jako L1 (level 1), další čísla
označují větší a pomalejší cache.

Tyto cache obsahují \uv{pracovní kopie} nedávno přečtených nebo zapsaných buněk
paměti.  Když CPU provádí instrukci pracující s obsahem RAM, pokusí se nejdříve
najít nebo přepsat hodnoty v cachích a pokud žádná cache neobsahuje kopii dané
buňky, a teprve když žádná rychlá cache tuto buňku neobsahuje, přistoupí k
pomalé hlavní paměti. Záznam z paměti se pak může propsat do cachí. Pokud jsou
všechna místa, na které jde daný záznam zacachovat, plná, určí takzvaná
\textsl{cache replacement policy} způsob, jak tuto situaci vyřešit. Obvykle
se z cache odebere ta položka, která byla použita naposledy. Tato cache
replacement policy se nazývá LRU (\textit{least recently used}).

Protože by bylo složité a paměťově náročné separátně cachovat každý byte paměti,
ukládají se do cachí pouze souvislé bloky nazývané \textsl{cache lines}.
Například architektura Core i7 má v cachích L1, L2 i L3 shodnou velikost cache
line 64 B.

Tato hierarchie cachí vznikly především protože rychlost CPU se vyvíjela rychleji,
než rychlost levných dynamických pamětí. Statické paměti sice umožňují rychlejší
operace, ale jsou také podstatně dražší. Aby CPU zbytečně nečekalo na RAM, může
díky cachím rychle pracovat s malým množstvím dat, ke kterým nedávno přistoupilo.
Mnoho algoritmů nepotřebuje v každém podprogramu zpracovávat velké množství dat,
a tedy se na nich projeví zrychlení díky cachím.\todo{konkretni materialy?}

Rychlost přístupu k datům v reálném počítači je tedy řízena dvěma faktory:
\begin{itemize}
\item
	\textit{Temporal locality}: Pokud byla data nedávno použita, jsou
	pravděpodobně uložena v některé z cachí. Čím dále v minulosti byla
	data naposledy použita, tím pomalejší bude další operace nad těmito
	daty.
\item
	\textit{Spatial locality}: Data se cachují po drobných blocích. Čtení
	z disků taktéž probíhá po blocích (obvykle velkých 4 kB), a kromě
	toho je mnohem rychlejší provádět na discích sekvenční čtení než čtení
	z náhodných míst. Obecně je rychlejší přistupovat k údajům uloženým
	blízko místa, které právě upravujeme.
\end{itemize}

\todo{kvuli tomu existuje EM model}

\chapter{Název první kapitoly}

\section{Název první podkapitoly v první kapitole}

\section{Název druhé podkapitoly v první kapitole}

\subsection{Ukázka}
\label{ssec:ukazka}

\begin{itemize}
\item Logo Matfyzu vidíme na obrázku~\ref{fig:mff}.
\item Tato subsekce má číslo~\ref{ssec:ukazka}.
\end{itemize}

červeno-černý (krátká),
strana 16--22 (střední),
$45-44$ (minus),
a~toto je --- jak se asi dalo čekat --- vložená věta ohraničená dlouhými pomlčkami.
(Všimněte si, že jsme za \verb|a| napsali vlnovku místo mezery: to aby se
tam nemohl rozdělit řádek.)

\newtheorem{theorem}{Věta}
\newtheorem*{define}{Definice}	% Definice nečíslujeme, proto "*"

\begin{define}
{\sl Strom} je souvislý graf bez kružnic.
\end{define}

\begin{theorem}
Tato věta neplatí.
\end{theorem}

\begin{proof}
Neplatné věty nemají důkaz.
\end{proof}

\begin{figure}
	\centering
	\includegraphics[width=30mm]{img/logo.eps}
	\caption{Logo MFF UK}
	\label{fig:mff}
\end{figure}

\begin{itemize}
\item Model externí paměti

	CPU + malá cache + velká externí paměť.
	Zanedbáme všechny cache kromě poslední.

	Block size $B$ (v ext. paměti i v cachi),
	velikost cache $M$, tedy $M/B$ cache lines.

	Registry, L1, L2, RAM, disk, síť, ...
	Typicky L1 cache line = L2 cache line = 64B.
	\cite{Vit}

	\begin{itemize}
	\item Základní datové struktury: queue, stack
	\end{itemize}

\item Co je to slovník, operace
	\begin{itemize}
	\item Příklady použití
	\item Rozšíření: successor, predecessor
		\begin{itemize}
		\item Příklady použití
		\end{itemize}
	\item Rozšíření: multiset
		\begin{itemize}
		\item Příklady použití
		\end{itemize}
	\end{itemize}

\item Triviální implementace slovníku
	\begin{itemize}
	\item Triviální pole (unrolled linked list?)
	\item Triviální přímá adresace pro malé klíče

\item Hashování
	\begin{itemize}
		\item Seznam prvků v kolizi
		\item Linear probing (open addressing)
		\item Double hashing
		\item Cuckoo hashing
	\end{itemize}

\item Vyhledávací stromy
	\begin{itemize}
		\item B-stromy (a 2-3-4 stromy)
		\item Červenočerné stromy
		\item Splay stromy (TODO: zkusit vymyslet N-arni splay stromy)
		\item Van Emde-Boas uspořádání
		\item T-stromy (?)
	\end{itemize}

\item Speciality pro vnější paměť
	\begin{itemize}
		\item Directory hashing (extensible hashing, linear hashing)
		\item Directoryless hashing methods
		\item Tree EM structures
	\end{itemize}


\item Popis prostředí, kde měřím
\end{itemize}

\include{zaver}

%%% Seznam použité literatury
\include{literatura}

%%% Tabulky v bakalářské práci, existují-li.
\chapwithtoc{Seznam tabulek}

%%% Použité zkratky v bakalářské práci, existují-li, včetně jejich vysvětlení.
\chapwithtoc{Seznam použitých zkratek}

%%% Přílohy k bakalářské práci, existují-li (různé dodatky jako výpisy programů,
%%% diagramy apod.). Každá příloha musí být alespoň jednou odkazována z vlastního
%%% textu práce. Přílohy se číslují.
\chapwithtoc{Přílohy}

\openright
\end{document}
