\section{Notes on optimization}
TODO: this primarily serves as a place to put my notes.
I may eventually put some of this into the final text.

\subsection{Inlining of van Emde Boas drilldown hot spots}
See commit \texttt{76d4a6c}. The main Van Emde Boas drilldown
methods (\texttt{drilldown\_go\_left} and \texttt{drilldown\_go\_right})
were previously compiled in a separate module. Making them inline
in the header file increased cache-oblivious B-tree performance
by approximately 10\%.

\subsection{Removing double walking in COBT insertions}
See commit \texttt{2fcdee5dd}.
\texttt{cob\_insert} needs to test that the inserted key doesn't exist
in the data structure yet. The first version performed this by calling
\texttt{cob\_find} and then finding the leaf to insert into, inserting
and fixing the van Emde Boas tree. While the code was very easy to understand,
it also duplicated the step of finding the correct leaf, since it used
to be done first by \texttt{cob\_find} and then by \texttt{cob\_insert}.

Changing the code to first look up the leaf and then to test for key presence
and possibly insert the new key-value pair increased performance.

Tested on \texttt{bin/experiments/performance -m 10M -a dict\_cobt}:
Before: 111.93 107.52 108.41 108.59 114.54, after:
99.95, 101.05, 102.63, 102.41, 100.96.
The speedup is about 8\%.

\subsection{Keeping stack for splaying in \texttt{splay\_tree.c}}
See commit \texttt{90f9247}.
\texttt{splay\_tree\_insert} previously first added a new node, and then called
a separate function to find the new node by its key and to splay it up.
Constructing a "splay stack" while inserting enhanced performance by about 12\%.
Original \texttt{bin/experiments/performance -m 10M -b 5 -a dict\_splay}:
13.770, 13.589, 13.868. After: 12.348, 12.172, 12.15.
